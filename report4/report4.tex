% vim: spelllang=uk
\documentclass[a4paper,12pt,notitlepage,pdftex]{scrreprt}
\usepackage{pdflscape}
\usepackage{cmap}
\usepackage[utf8]{inputenc}
\usepackage[english,ukrainian]{babel}
\usepackage[T2A]{fontenc}
\usepackage{a4wide}

\usepackage[pdftex]{graphicx}

\pdfcompresslevel=9 % сжимать PDF
\usepackage{pdflscape} % для возможности альбомного размещения некоторых страниц
\usepackage[pdftex]{hyperref}
% настройка ссылок в оглавлении для pdf формата
\hypersetup{unicode=true,
    pdftitle={Лабораторна робота №4},
    pdfauthor={Кулик Ольга},
    pdfcreator={pdflatex},
    pdfsubject={},
    pdfborder    = {0 0 0},
    bookmarksopen,
    bookmarksnumbered,
    bookmarksopenlevel = 2,
    pdfkeywords={},
    colorlinks=true, % установка цвета ссылок в оглавлении
    citecolor=black,
    filecolor=black,
    linkcolor=black,
    urlcolor=blue
}

\usepackage{amsmath}
\usepackage{amssymb}
\usepackage{listings}
\lstloadlanguages{[Visual]Basic,C++}
\lstset{language=[Visual]Basic,frame=tb,basicstyle=\small,commentstyle=\itshape,extendedchars=false}

\usepackage{multicol}

\begin{document}
\begin{titlepage}
    \begin{center}
        \MakeUppercase{Міністерство освіти і науки України}\\
        \MakeUppercase{Національний технічний університет України}\\
        \MakeUppercase{,,Київський політехнічний інститут''}\\
        \vspace*{2em}

        Факультет прикладної математики\\
        Кафедра прикладної математики

        \vfill

        \MakeUppercase{Генератор проміжного коду}\\
        \vspace*{1em}

        Звіт з лабораторної роботи №4\\
        з дисципліни:\\
        <<Системи автоматизації програмування>>
    \end{center}

    \vfill
    \hfill\begin{minipage}{0.4\textwidth}
        Виконала студентка\\
        групи КМ-31м\\
        Кулик~О.\,О.
    \end{minipage}

    \vfill
    \begin{center}
        Київ\\
        2014
    \end{center}
\end{titlepage}

\tableofcontents

\chapter{Постановка задачі}
\label{chap:first}
    Для виконання лабораторної роботи необхідно розробити схему перекладу для основних синтаксичний елементів мови
    (декларацій, операторів, блоків і програми в цілому), визначити таблицю базових операцій проміжного коду та
    написати відповідний програмний модуль.

    В якості схеми перекладу слід використовувати ПОЛІЗ (інверсний польський запис).
\chapter{Схема перекладу в проміжний код}
    Генерацію проміжного коду для програми в цілому можна представити як суперпозицію генерацій коду для базових
    синтаксичний елементів.

    \section{Базові бінарні операції}
        Будь-яка бінарна операція \verb"a op b" представляється у вигляді \verb"a b op".
        В якості \verb"op" можуть виступати наступні оператори: \verb":=, ==, !=, <, >, +, -, *, /".
    \section{Звертання до елементу масива}
        Звертання до елементу масиву \verb"a[ i ]" перетворюється у \verb"a i ]".
        Слід зазначити, що оскільки в підмножині мови, що розглядається, можливі тільки одновимірні масиви, то не
        потрібно перед \verb"]" зазначати кількість індексів, що треба робити в загальному випадку.
    \section{Оператори введення та виведення}
        Оператор введення в підмножині мови, що розглядається, не приймає жодних параметрів, а тому в проміжному коду
        він має вигляд лише одної команди: \verb"READ".

        Оператор виведення приймає єдиний параметр, тому в проміжний код він перетворюється за наступною схемою
        \verb"TextWindow.Write( a )" $\Rightarrow$ \verb$ a WRITE$.
    \section{Декларації мітки та безумовний перехід}
        Декларація мітки має вигляд числа з двокрапкою: \verb$m:$.

        Безумовний перехід на мітки \verb$m$ здійснюється за допомогою команди \verb$m BR$.
    \section{Умовний оператор}
        Оператор розгалуження \verb$If <uslovie> Then ... Else ... EndIf$ перетворюється за наступною схемою:
        \verb$<uslovie> m1 ifFalse ... m2 BR m1: ... m2:$.

        Команда \verb"ifFalse" перейде на мітку \verb$m1$ якщо \verb$<uslovie>$ не є істиною.
    \section{Оператор циклу}
        Оператор циклу \verb$While <uslovie> ... EndWhile$ представляється у проміжному коді наступним чином:
        \verb$m1: <uslovie> m2 ifFalse ... m1 BR m2:$.
    \section{Оператор виклику процедури}
        Оскільки в підмножині мови, що розглядається, процедури не приймають жодних параметрів, то виклик процедури
        має вигляд \verb"WriteArray()" $\Rightarrow$ \verb$WriteArray Fn$.
    \section{Блоки коду, декларація процедур}
        Блоки коду мають вигляд \verb$BLOCK ... BLCKEND$.

        Декларація процедур має вигляд \verb$<procname>: BLOCK ... BLCKEND$.
\chapter{Організація генерації коду}
    Модуль генерації коду використовує синтаксичне дерево, що було побудовано як результат роботи синтаксичного
    аналізатора після процедури семантичного аналізу.

    Модуль генерації здійснює обхід дерева у глибину й для кожної вершини використовує правила, що були описані у
    попередньому параграфі для перетворення відповідних вершин.
\chapter{Тестовий приклад}
\label{chap:fourth}
        \lstinputlisting{../test.ba}
        \paragraph{Проміжний код:}
        \begin{verbatim}
BEGIN_OF_PROGRAM
WriteArray:
BLOCK
k  0  :=
1:  k  4  <
2  ifFalse
BLOCK
k  k  1  +  :=
 a  k  ]  Write
", "  Write
BLCKEND
1  BR
2:
a  k  ]  Write
BLCKEND

i  0  :=
3:  i  5  <
4  ifFalse
BLOCK
i  i  1  +  :=
i  3  !=
5  ifFalse
BLOCK
a  i  ]  i  i  *  :=
BLCKEND
6  BR
5:
BLOCK
a  i  ]  42  :=
BLCKEND
6:
BLCKEND
3  BR
4:
WriteArray  Fn
END_OF_PROGRAM
\end{verbatim}

\chapter{Висновки}
\label{chap:concl}
    Під час виконання лабораторної роботи було виявлено схеми перекладу основних синтаксичних конструкцій у проміжний
    код у формі ПОЛІЗу.
    Сама генерація використовує результати другої лабораторної роботи, а саме синтаксичне дерево.

    Оскільки в підмножині мови, що розглядається відсутні багатовимірні масиви та параметри процедур, проміжний код
    має дещо спрощений вигляд у порівнянні з тим, що був викладений на лекціях.

    Також можна зазначити, що для генерації проміжного кода потрібно створювати мітки, а тому, щоб не було конфліктів
    з існуючими мітками, потрібно мати механізм для пошуку вільного ідентифікатора мітки.

\chapter*{Додаток. Текст програми}
    \lstset{language=C++,basicstyle=\scriptsize}
        \lstinputlisting{../main_window.cxx}
\end{document}
