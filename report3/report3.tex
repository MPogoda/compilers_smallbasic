% vim: spelllang=uk
\documentclass[a4paper,12pt,notitlepage,pdftex]{scrreprt}
\usepackage{pdflscape}
\usepackage{cmap}
\usepackage[utf8]{inputenc}
\usepackage[english,ukrainian]{babel}
\usepackage[T2A]{fontenc}
\usepackage{a4wide}

\usepackage[pdftex]{graphicx}

\pdfcompresslevel=9 % сжимать PDF
\usepackage{pdflscape} % для возможности альбомного размещения некоторых страниц
\usepackage[pdftex]{hyperref}
% настройка ссылок в оглавлении для pdf формата
\hypersetup{unicode=true,
    pdftitle={Лабораторна робота №2},
    pdfauthor={Кулик Ольга},
    pdfcreator={pdflatex},
    pdfsubject={},
    pdfborder    = {0 0 0},
    bookmarksopen,
    bookmarksnumbered,
    bookmarksopenlevel = 2,
    pdfkeywords={},
    colorlinks=true, % установка цвета ссылок в оглавлении
    citecolor=black,
    filecolor=black,
    linkcolor=black,
    urlcolor=blue
}

\usepackage{amsmath}
\usepackage{amssymb}
\usepackage{listings}
\lstloadlanguages{[Visual]Basic,C++}
\lstset{language=[Visual]Basic,frame=tb,basicstyle=\small,commentstyle=\itshape,extendedchars=false}

\usepackage{multicol}

\begin{document}
\begin{titlepage}
    \begin{center}
        \MakeUppercase{Міністерство освіти і науки України}\\
        \MakeUppercase{Національний технічний університет України}\\
        \MakeUppercase{,,Київський політехнічний інститут''}\\
        \vspace*{2em}

        Факультет прикладної математики\\
        Кафедра прикладної математики

        \vfill

        \MakeUppercase{Побудова таблиці символів}\\
        \vspace*{1em}

        Звіт з лабораторної роботи №3\\
        з дисципліни:\\
        <<Системи автоматизації програмування>>
    \end{center}

    \vfill
    \hfill\begin{minipage}{0.4\textwidth}
        Виконала студентка\\
        групи КМ-31м\\
        Кулик~О.\,О.
    \end{minipage}

    \vfill
    \begin{center}
        Київ\\
        2014
    \end{center}
\end{titlepage}

\tableofcontents

\chapter{Постановка задачі}
\label{chap:first}
    Для виконання лабораторної роботи необхідно розглянути, організувати й побудувати таблицю символів до
    синтаксичного аналізатора для підмножини мови програмування \texttt{SmallBasic}, що була описана в попередній
    лабораторній роботі.

\chapter{Організація таблиці символів та її компонентів}
\label{chap:second}
    Оскільки у підмножині мови, з якою має працювати семантичний аналізатор, майже немає областей видимості змінних
    (єдиний механізм для цього --- доданий штучний шляхом до мови оператор \verb'Dim'), то основні зберігаються у
    глобальній таблиці, яка містить наступні елементи:
    \begin{itemize}
        \item Таблиця символів, що були використані (це може бути змінна, масив чи процедура).
        \item Таблиця символів, що були задекларовані.
            Якщо після виконання програми перша таблиця має символи, що не містяться у другій --- це означає
            семантичну помилку.
        \item Таблиця міток (використаних й створених).
        \item Посилання на <<локальну таблицю>>, яка має наступний вигляд:
            \begin{itemize}
                \item Імена змінних, що були задекларовані за допомогою оператору \verb'Dim' у поточному блоці коду.
                \item Посилання на вкладені локальні таблиці.
            \end{itemize}
    \end{itemize}
    \section{Обраний метод пошуку символів}
    \label{sec:formal}
        В даній роботі використовується пошук символів за допомогою хеш"=таблиці.
        Оскільки процедури у мові мають характер макросів, то неможливо заздалегідь знати, чи буде задекларований той
        чи інший символ.
        Через це кожен використаний символ зберігається у відповідній таблиці до того часу, як він не буде
        задекларований.

        Семантичний аналіз перевіряє, що один й той самий символ не використовується для різних категорій (змінна,
        масив чи процедура).

        Після аналізу програми буде видно, які символи не були задекларовані.
\chapter{Метод побудови таблиці символів}
    Під час використання синтаксичного аналізатора, що був розроблений під час попередньої лабораторної роботи можна
    виділити скінченну кількість нетермінальних символів, під час розбору яких треба використовувати таблицю символів,
    а саме:
    \begin{itemize}
        \item \verb'{new-identifier}'
        \item \verb'{dim-expr}'
        \item \verb'{id}'
        \item \verb'{sub-stmt}'
    \end{itemize}

    Під час розбору цих вузлів синтаксичного дерева виконуються семантичні процедури, що будують таблицю символів і
    виконують наступні функції:
    \begin{itemize}
        \item Перевірка, що один й той самий символ використовуються завжди в одному із трьох типів (масив, змінна чи
            процедура).
        \item Перевірка, що процедура не оголошується більше одного разу.
        \item Перевірка, що ім’я локальної змінної не перекриває глобальну змінну.
        \item Перевірка, що кожен використаний символ був оголошений.
        \item Перевірка, що кожна використана мітка була оголошена.
    \end{itemize}
\chapter{Тестові приклади}
\label{chap:fourth}
    \section{Коректний приклад}
    \label{sec:ex1}
        \lstinputlisting{../test.ba}
        \begin{verbatim}
Available global definitions:
  WriteArray : Procedure
  a : Array
  i : Variable
  k : Variable
Available labels:
Available new local variables: {{NONE}}
  Available new local variables: {{NONE}}
    Available new local variables: {{NONE}}
  Available new local variables: {{NONE}}
    Available new local variables: {{NONE}}
    Available new local variables: {{NONE}}
\end{verbatim}

    \section{Приклад з семантичною помилкою}
    \label{sec:ex2}
        \begin{lstlisting}
            a = b + c
        \end{lstlisting}
        \begin{verbatim}
Available global definitions:
  a : Variable
NOT DECLARED, BUT USED DECLARATIONS:
  b : Variable
  c : Variable
Available labels:
Available new local variables: {{NONE}}
\end{verbatim}

\chapter{Висновки}
\label{chap:concl}
    Для семантичної коректності програми необхідно знати складові програми, характеристики ідентифікаторів, тощо.
    Семантичний контроль задля ефективності був вбудований у фазу синтаксичного аналізу: під час побудови дерева
    синтаксичного розбору в залежності від типу вузла виконуються певні семантичні процедурі, що й будують таблицю
    символів та перевіряють семантичну коректність програми.
\chapter*{Додаток. Текст програми}
    \lstset{language=C++,basicstyle=\scriptsize}
        \lstinputlisting{../program_struct.h}
\end{document}
