% vim: spelllang=uk
\documentclass[a4paper,12pt,notitlepage,pdftex]{scrreprt}
\usepackage{pdflscape}
\usepackage{cmap}
\usepackage[utf8]{inputenc}
\usepackage[english,ukrainian]{babel}
\usepackage[T2A]{fontenc}
\usepackage{a4wide}

\usepackage[pdftex]{graphicx}

\pdfcompresslevel=9 % сжимать PDF
\usepackage{pdflscape} % для возможности альбомного размещения некоторых страниц
\usepackage[pdftex]{hyperref}
% настройка ссылок в оглавлении для pdf формата
\hypersetup{unicode=true,
    pdftitle={Лабораторна робота №2},
    pdfauthor={Кулик Ольга},
    pdfcreator={pdflatex},
    pdfsubject={},
    pdfborder    = {0 0 0},
    bookmarksopen,
    bookmarksnumbered,
    bookmarksopenlevel = 2,
    pdfkeywords={},
    colorlinks=true, % установка цвета ссылок в оглавлении
    citecolor=black,
    filecolor=black,
    linkcolor=black,
    urlcolor=blue
}

\usepackage{amsmath}
\usepackage{amssymb}
\usepackage{listings}
\lstloadlanguages{[Visual]Basic,C++}
\lstset{language=[Visual]Basic,frame=tb,basicstyle=\small,commentstyle=\itshape,extendedchars=false}

\usepackage{multicol}

\begin{document}
\begin{titlepage}
    \begin{center}
        \MakeUppercase{Міністерство освіти і науки України}\\
        \MakeUppercase{Національний технічний університет України}\\
        \MakeUppercase{,,Київський політехнічний інститут''}\\
        \vspace*{2em}

        Факультет прикладної математики\\
        Кафедра прикладної математики

        \vfill

        \MakeUppercase{Побудова синтаксичного аналізатора}\\
        \vspace*{1em}

        Звіт з лабораторної роботи №2\\
        з дисципліни:\\
        <<Системи автоматизації програмування>>
    \end{center}

    \vfill
    \hfill\begin{minipage}{0.4\textwidth}
        Виконала студентка\\
        групи КМ-31м\\
        Кулик~О.\,О.
    \end{minipage}

    \vfill
    \begin{center}
        Київ\\
        2014
    \end{center}
\end{titlepage}

\tableofcontents

\chapter{Постановка задачі}
\label{chap:first}
    Для виконання лабораторної роботи необхідно побудувати синтаксичний аналізатор (СА) для підмножини мови
    програмування \texttt{SmallBasic}, що була описана в попередній лабораторній роботі.

    Синтаксичний аналізатор повинен використовувати ліву рекурсію (LL).
    \textit{LL аналізатор} --- це алгоритм синтаксичного аналізу методом рекурсивного спуску для підмножини
    контекстно"=вільних граматик.
    Він обробляє вхід зліва направо (тому перша буква означає Left), та будує ліворекурсивне виведення рядка.
\chapter{Технічне завдання}
\label{chap:second}
    \section{Формальне визначення вихідної граматики}
    \label{sec:formal}
        Вихідна граматика формально описується як четвірка:
        \begin{equation}
            \label{eq:grammar}
            G = \left( N, \Sigma, P, S \right)
        \end{equation}
        , де
        \begin{itemize}
            \item $N$ --- алфавіт нетермінальних символів, наведений у на сторінці~\pageref{para:nonterm};
            \item $\Sigma$ --- алфавіт термінальних символів, який містить лексеми, що були здобуті під час лексичного
                аналізу; наведені на сторінці~\pageref{para:term};
            \item $P$ --- множина правил граматики, наведена на сторінці~\pageref{para:rules};
            \item $S$ --- початковий символ граматики (дорівнює \verb'<start>').
        \end{itemize}

        \paragraph{Нетермінальні символи граматики}
        \label{para:nonterm}
        \begin{multicols}{3}
            \begin{enumerate}
                \item \verb'{start}'
                \item \verb'{all-stmt}'
                \item \verb'{all-stmts}'
                \item \verb'{more-all-stmts}'
                \item \verb'{sub-stmt}'
                \item \verb'{stmts}'
                \item \verb'{more-stmts}'
                \item \verb'{lstmt}'
                \item \verb'{stmt}'
                \item \verb'{label-def}'
                \item \verb'{dim-expr}'
                \item \verb'{assignment}'
                \item \verb'{id}'
                \item \verb'{array-index}'
                \item \verb'{operand-int}'
                \item \verb'{rightside}'
                \item \verb'{read-stmt}'
                \item \verb'{int-expr}'
                \item \verb'{operator-int}'
                \item \verb'{logic}'
                \item \verb'{logic-int}'
                \item \verb'{compare-int}'
                \item \verb'{logic-equals}'
                \item \verb'{logic-bool}'
                \item \verb'{operand-bool}'
                \item \verb'{logic-str}'
                \item \verb'{operand-str}'
                \item \verb'{if-stmt}'
                \item \verb'{else-part}'
                \item \verb'{goto-stmt}'
                \item \verb'{while-stmt}'
                \item \verb'{write-stmt}'
                \item \verb'{writeable}'
                \item \verb'{sub-call-stmt}'
            \end{enumerate}
        \end{multicols}

        \paragraph{Термінальні символи граматики}
        \label{para:term}
        \begin{multicols}{3}
            \begin{enumerate}
                \item \verb'('
                \item \verb')'
                \item \verb'+'
                \item \verb'-'
                \item \verb'*'
                \item \verb'/'
                \item \verb':'
                \item \verb'['
                \item \verb']'
                \item \verb'='
                \item \verb'<'
                \item \verb'>'
                \item \verb'Sub'
                \item \verb'TextWindow.Write'
                \item \verb'While'
                \item \verb'EndWhile'
                \item \verb'Goto'
                \item \verb'Else'
                \item \verb'If'
                \item \verb'Then'
                \item \verb'EndIf'
                \item \verb'TextWindow.Read'
                \item \verb'Dim'
                \item \verb'Sub'
                \item \verb'EndSub'
                \item \verb'{int-constant}' --- будь-яке ціле число.
                \item \verb'{bool-constant}' --- \verb'"true"' чи \verb'"false"'.
                \item \verb'{string-constant}' --- будь-яка строкова константа.
                \item \verb'{identifier}' --- будь-який ідентифікатор.
                \item \verb'{newline}' --- кінець рядка.
                \item $\varepsilon$ --- порожній символ.
            \end{enumerate}
        \end{multicols}

        \paragraph{Правила вихідної граматики}
        \label{para:rules}
            Множина правил граматики подається у вигляді $ A \rightarrow B$.
            \begin{enumerate}
                \item \verb'{start}' $\rightarrow$ \verb'{all-stmts}'
                \item \verb'{all-stmt}' $\rightarrow$ \verb'{sub-stmt}'
                \item \verb'{all-stmt}' $\rightarrow$ \verb'{lstmt}'
                \item \verb'{all-stmts}' $\rightarrow$ \verb'{all-stmt} {more-all-stmts}'
                \item \verb'{more-all-stmts}' $\rightarrow$ \verb'{all-stmts}'
                \item \verb'{more-all-stmts}' $\rightarrow \varepsilon$
                \item \verb'{sub-stmt}' $\rightarrow$ \verb'Sub {identifier} {newline} {stmts} EndSub {newline}'
                \item \verb'{stmts}' $\rightarrow$ \verb'{lstmt} {more-stmts}'
                \item \verb'{more-stmts}' $\rightarrow$ \verb'{stmts}'
                \item \verb'{more-stmts}' $\rightarrow \varepsilon$
                \item \verb'{lstmt}' $\rightarrow$ \verb'{label-def} {stmt}'
                \item \verb'{lstmt}' $\rightarrow$ \verb'{label-def} {newline}'
                \item \verb'{stmt}' $\rightarrow$ \verb'{dim-expr}'
                \item \verb'{stmt}' $\rightarrow$ \verb'{assignment}'
                \item \verb'{stmt}' $\rightarrow$ \verb'{if-stmt}'
                \item \verb'{stmt}' $\rightarrow$ \verb'{goto-stmt}'
                \item \verb'{stmt}' $\rightarrow$ \verb'{while-stmt}'
                \item \verb'{stmt}' $\rightarrow$ \verb'{write-stmt}'
                \item \verb'{stmt}' $\rightarrow$ \verb'{sub-call-stmt}'
                \stepcounter{enumi}
                \item \verb'{label-def}' $\rightarrow$ \verb'{int_constant} :'
                \item \verb'{label-def}' $\rightarrow \varepsilon$
                \item \verb'{dim-expr}' $\rightarrow$ \verb'Dim  {identifier} {newline}'
                \item \verb'{assignment}' $\rightarrow$ \verb'{id} = {rightside} {newline}'
                \item \verb'{id}' $\rightarrow$ \verb'{identifier} {array-index}'
                \item \verb'{id}' $\rightarrow$ \verb'{identifier}'
                \item \verb'{array-index}' $\rightarrow$ \verb'[ {operand-int} ]'
                \item \verb'{operand-int}' $\rightarrow$ \verb'{id}'
                \item \verb'{operand-int}' $\rightarrow$ \verb'{int-constant}'
                \item \verb'{rightside}' $\rightarrow$ \verb'{read-stmt}'
                \item \verb'{rightside}' $\rightarrow$ \verb'{int-expr}'
                \item \verb'{rightside}' $\rightarrow$ \verb'{logic}'
                \item \verb'{rightside}' $\rightarrow$ \verb'{string-constant}'
                \item \verb'{read-stmt}' $\rightarrow$ \verb'TextWindow.Read ( )'
                \item \verb'{int-expr}' $\rightarrow$ \verb'{operand-int} {operator-int} {operand-int}'
                \item \verb'{int-expr}' $\rightarrow$ \verb'{operand-int}'
                \item \verb'{operator-int}' $\rightarrow$ \verb'+'
                \item \verb'{operator-int}' $\rightarrow$ \verb'-'
                \item \verb'{operator-int}' $\rightarrow$ \verb'/'
                \item \verb'{operator-int}' $\rightarrow$ \verb'*'
                \item \verb'{logic}' $\rightarrow$ \verb'{logic-int}'
                \item \verb'{logic}' $\rightarrow$ \verb'{logic-bool}'
                \item \verb'{logic}' $\rightarrow$ \verb'{logic-str}'
                \item \verb'{logic-int}' $\rightarrow$ \verb'{operand-int} {compare-int} {operand-int}'
                \item \verb'{compare-int}' $\rightarrow$ \verb'<'
                \item \verb'{compare-int}' $\rightarrow$ \verb'>'
                \item \verb'{compare-int}' $\rightarrow$ \verb'{logic-equals}'
                \item \verb'{logic-equals}' $\rightarrow$ \verb'='
                \item \verb'{logic-equals}' $\rightarrow$ \verb'<>'
                \item \verb'{logic-bool}' $\rightarrow$ \verb'{operand-bool} {logic-equals} {operand-bool}'
                \item \verb'{logic-bool}' $\rightarrow$ \verb'{operand-bool}'
                \item \verb'{operand-bool}' $\rightarrow$ \verb'{id}'
                \item \verb'{operand-bool}' $\rightarrow$ \verb'{bool-constant}'
                \item \verb'{logic-str}' $\rightarrow$ \verb'{operand-str} {logic-equals} {operand-str}'
                \item \verb'{operand-str}' $\rightarrow$ \verb'{id}'
                \item \verb'{operand-str}' $\rightarrow$ \verb'{string_constant}'
                \item \verb'{if-stmt}' $\rightarrow$ \verb'If{logic}Then{newline}{stmts}{else-part}EndIf{newline}'
                \item \verb'{else-part}' $\rightarrow$ \verb'Else {newline} {stmts}'
                \item \verb'{else-part}' $\rightarrow \varepsilon$
                \item \verb'{goto-stmt}' $\rightarrow$ \verb'Goto {int-constant} {newline}'
                \item \verb'{while-stmt}' $\rightarrow$ \verb'While {logic} {newline} {stmts} EndWhile {newline}'
                \item \verb'{write-stmt}' $\rightarrow$ \verb'TextWindow.Write ( {writeable} ) {newline}'
                \item \verb'{writeable}' $\rightarrow$ \verb'{int-expr}'
                \item \verb'{writeable}' $\rightarrow$ \verb'{logic}'
                \item \verb'{writeable}' $\rightarrow$ \verb'{string-constant}'
                \item \verb'{sub-call-stmt}' $\rightarrow$ \verb'{identifier} ( ) {newline}'
            \end{enumerate}
\chapter{Опис роботи програми та її логічної структури}
\label{chap:third}
    В даній роботі синтаксичний аналіз виконується за допомогою \texttt{LL(1)} метода.
    В якості вхідних символів граматики використовуються лексеми, отримані за допомогою лексичного аналізатору,
    розробленого в попередній лабораторній роботі.

    Синтаксичний аналіз проходить одразу після коректного завершення роботи лексичного аналізатора.

    Алгоритм синтаксичного розбору використовує сконструйовану заздалегідь таблицю переходів.

    Результатом роботи синтаксичного аналізатора є ліворекурсивне виведення рядка чи повідомлення про помилку.
\chapter{Тестові приклади}
\label{chap:fourth}
    \section{Коректний приклад}
    \label{sec:ex1}
        \lstinputlisting{../test.ba}
        Цей вихідний текст має наступний ланцюжок розбору:
        1 $\rightarrow$ 4 $\rightarrow$ 2 $\rightarrow$ 7 $\rightarrow$ 8 $\rightarrow$ 11 $\rightarrow$ 22
          $\rightarrow$ 14 $\rightarrow$ 24 $\rightarrow$ 26 $\rightarrow$ 31 $\rightarrow$ 36 $\rightarrow$ 29
          $\rightarrow$ 9 $\rightarrow$ 8 $\rightarrow$ 11 $\rightarrow$ 22 $\rightarrow$ 17 $\rightarrow$ 61
          $\rightarrow$ 41 $\rightarrow$ 44 $\rightarrow$ 28 $\rightarrow$ 26 $\rightarrow$ 45 $\rightarrow$ 29
          $\rightarrow \dots$
          $\rightarrow$ 8 $\rightarrow$ 11 $\rightarrow$ 22 $\rightarrow$ 14 $\rightarrow$ 24 $\rightarrow$ 26
          $\rightarrow$ 35 $\rightarrow$ 28 $\rightarrow$ 26 $\rightarrow$ 40 $\rightarrow$ 28 $\rightarrow$ 26
          $\rightarrow$ 10 $\rightarrow$ 58 $\rightarrow$ 8 $\rightarrow$ 11 $\rightarrow$ 22 $\rightarrow$ 14
          $\rightarrow$ 24 $\rightarrow$ 25 $\rightarrow$ 27 $\rightarrow$ 28 $\rightarrow$ 26 $\rightarrow$ 31
          $\rightarrow$ 36 $\rightarrow$ 29 $\rightarrow$ 10 $\rightarrow$ 10 $\rightarrow$ 5 $\rightarrow$ 4
          $\rightarrow$ 3 $\rightarrow$ 11 $\rightarrow$ 22 $\rightarrow$ 19 $\rightarrow$ 66 $\rightarrow$ 6
    \section{Приклад з синтаксичною помилкою}
    \label{sec:ex2}
        \begin{lstlisting}
            a = Dim
        \end{lstlisting}
\chapter{Висновки}
\label{chap:concl}
    Як результат фази лексичного аналізу будується лексична згортка, по якій працює синтаксичний аналізатор.
    В результаті отримуємо чи синтаксичний розбір тексту програми, чи повідомлення про помилку.

    В даній роботі був реалізований синтаксичний аналізатор, що використовує LL(1) граматику, показані його основні
    можливості, приведені коректні й некоректні приклади.
\chapter*{Додаток. Текст програми}
    \lstset{language=C++,basicstyle=\scriptsize}
    \lstinputlisting{../syntax.h}
    \lstinputlisting{../syntax.cxx}
\end{document}
